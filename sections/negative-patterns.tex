\section{Negative Patterns}
To use when we are dealing with algorithm that are unlikely to be solved with conventional tecniques.\\\\
\emph{Objective}: Classify problems according to those that can be solved in polynomial-time and those that cannot. Given a black box or a Turing machine, does it stops in k steps? Unfortunately huge number of fundamental problems have defied classification for decades.

\subsection{Polinomial Time Reduction}
Suppose we could solve X in polynomial-time. What else could we solve in polynomial time?

\[ X \leq_{p} Y\]

\emph{Formal Definition}: Problem X polynomial reduces to problem Y if arbitrary instances of problem X can be solved using: polynomial number of standard computational steps + polynomial number of calls to oracle that solves problem Y.\\

\emph{Practical Definition}: Problem X polynomial reduces to problem Y, If it's possible to transform the X problem instance into Y instance in polynomial time.

\begin{itemize}
    \item{If $X \leq_{p} Y$ and Y can be solved in polynomial-time, then X can also be solved in polynomial time}
    \item{If $X \leq_{p} Y$ and X cannot be solved in polynomial-time, then Y cannot be solved in polynomial time}
    \item{If $X \leq_{p} Y$ and $Y \leq_{p} X$ then $X =_{p} Y$}
\end{itemize}

\subsection{Reduction by Equivalence}

\emph{The indipendent set Problem}:\\

Given a graph G = (V, E) and an integer k, is there a subset of vertices $S \subseteq V$ such that $|S| \geq k$, and for each edge at most one of its endpoints is in S?

\begin{figure}[H]
    \centering
    \includegraphics[width=0.6\textwidth ]{indipendent}
    \caption{An instance of Indipendent Set Problem}
\end{figure}

\emph{The vertex set Cover Problem}:\\

Given a graph G = (V, E) and an integer k, is there a subset of vertices $S \subseteq V$ such that $|S| \leq k$, and for each edge, at least one of its endpoints is in S?

\begin{figure}[H]
    \centering
    \includegraphics[width=0.6\textwidth ]{vertexCover}
    \caption{An instance of Vertex Cover Problem}
\end{figure}

\begin{claim}
    VERTEX-COVER $≡_{P}$ INDEPENDENT-SET.
\end{claim}\\

\begin{proof}
    We show S is an independent set if V − S is a vertex cover.\\

    Let S be any independent set.
    \begin{itemize}

        \item{Consider an arbitrary edge (u, v).}
        \item{S independent ⇒ $u \notin S \; or \;v \notin S$ }
        \item{Thus, V − S covers (u, v)}.
    \end{itemize}

    Let V − S be any vertex cover.
    \begin{itemize}
        \item{Consider two nodes $u \in S$ and $v \in S$.}
        \item{Observe that (u, v) $\notin$ E since V − S is a vertex cover }
        \item{Thus, no two nodes in S are joined by an edge, S is an indipendent set}.
    \end{itemize}

\end{proof}

\subsection{Reduction from a special case to a general case}

\emph{Set Cover}:
Given a set U of elements, a collection $S_{1}, S_{2}, . . . , S_{m}$ of subsets of U, and an integer k, does there exist a collection of $\leq k$ of these sets whose union is equal to U?

\begin{figure}[H]
    \centering
    \includegraphics[width=0.3\textwidth ]{setCover}
    \caption{An instance of Set Cover Problem}
\end{figure}

\begin{claim}
    $VERTEX-COVER \leq_{p} SET-COVER$, this is also known as Karp reduction.
\end{claim}\\

\begin{proof}
    Given a VERTEX-COVER instance G = (V, E), k, we construct a set cover instance whose size equals the size of the vertex cover instance.
\end{proof}

\begin{figure}[H]
    \centering
    \includegraphics[width=0.6\textwidth ]{vertexSet}
    \caption{Transformation from Vertex-Cover to Set-Cover}
\end{figure}

\subsection{Reduction via Gadgets}

\begin{itemize}
    \item{Literal: a Boolean variable or its negation: $x_{i} \; or x_{i}$}
    \item{Clause: A disjunction of literals: $C_{j} =x_{1} \land x_{2}  \land x_{3}$}
    \item{Conjunctive normal form:A propositional formula Φ that is the conjunction of clauses.$ Φ= C1 \lor C2 \lor C3 \lor C4$}
\end{itemize}

\begin{figure}[H]
    \centering
    \includegraphics[width=0.6\textwidth ]{sat}
\end{figure}

\begin{figure}[H]
    \centering
    \includegraphics[width=0.6\textwidth ]{indsat}
\end{figure}

\clearpage
